\documentclass[11pt,a4paper]{article}
\usepackage[utf8]{inputenc}
\usepackage[spanish]{babel}
\usepackage{amsmath}
\usepackage{amsfonts}
\usepackage{amssymb}
\usepackage{graphicx}
\usepackage{caption}
\captionsetup[table]{name=Tabla}
\usepackage{float}
\usepackage[left=2cm,right=2cm,top=2cm,bottom=2cm]{geometry}
\author{Marina Esgueva Ruiz}
\title{frontera 0 edp 3}
\begin{document}
\begin{table}[ht]
\begin{center}
\caption{EDP Wertz. RBF: imq. Condiciones de frontera igualadas a cero.}
\begin{tabular}{|c|c|c|c|c|}
\hline
 \multicolumn{2}{|c|}{ }& EDP original&Frontera 0 &  \\
 \hline
 $N$ & $\epsilon$ & $ECM$  & $ECM$ & Condicionamiento \\
 \hline 
 9 & 0.1 & 6.8318e-01 & 2.3103e-01 & 3.4989e+09 \\
 25 & 0.5 & 1.1637e-01 & 3.0674e-02 & 1.7687e+09 \\
 81 & 0.3 & 1.5598e-03 & 5.3217e-03 & 1.4280e+19 \\
 169 & 0.7 & 8.8125e-04 & 9.0307e-04 & 3.0855e+18 \\
 289 & 0.9 & 2.1364e-05 &  1.8145e-05 & 6.0151e+18 \\
 1089 & 2.1& 1.0772e-05& 2.2851e-06 & 2.2353e+20 \\
 \hline 
 
\end{tabular}

\end{center}
\end{table}
\end{document}