\documentclass[11pt,a4paper]{article}
\usepackage[utf8]{inputenc}
\usepackage[spanish]{babel}
\usepackage{amsmath}
\usepackage{amsfonts}
\usepackage{amssymb}
\usepackage{float}
\usepackage{graphicx}
\usepackage[left=2cm,right=2cm,top=2cm,bottom=2cm]{geometry}
\author{Marina Esgueva Ruiz}
\usepackage{caption}
\captionsetup[table]{name=Tabla}
\title{Variación del parámetro de forma}
\begin{document}
\begin{table}
\begin{center}
\caption{RBF:IMQ.  EDP 1. Búsqueda del parámetro de forma óptimo (izquierda) y búsqueda del parámetro de forma óptimo en la frontera fijandolo en el interior (derecha).}
\begin{tabular}{|c|ccc|ccc|}
\hline 
N & $\epsilon$ optimo & ECM & Condicionamineto & $\epsilon$ frontera & ECM & Condicionamiento \\ 
\hline 
9 & 0.7 & 8.5360e-03 & 2.3924e+03 & 0.5 & 6.2404e-03 & 1.0794e+04 \\ 
25 & 0.7& 1.1793e-03&2.0287e+07& 0.4& 1.1748e-03&1.9829e+08 \\
81& 0.3& 4.0351e-06& 1.4280e+19& 0.3& 4.0351e-06 & 1.4280e+19\\
169& 0.7&7.1329e-07& 3.0855e+18& 0.5 & 1.3167e-07 & 4.0733e+18 \\
289&0.9 & 1.3287e-07&  6.0151e+18& 0.8 & 5.9982e-08& 8.7037e+18\\
1089&2.1& 1.0321e-07& 2.2353e+20& 1& 1.3369e-08&1.3723e+21\\
\hline
\end{tabular}
\label{edp1} 
\end{center}
\end{table}

\begin{table}
\begin{center}
\caption{RBF: IMQ. EDP 1. Variación del parámetro en torno a la media fijada en la Tabla \ref{edp1}. $\epsilon_j=\bar{\epsilon} \frac{1}{\sqrt{1+K(-1)^j}}$. $K=0.3$ en el interior y $K=0.5$ en la frontera.  }
\begin{tabular}{|c|cc|}
\hline
$N$ & ECM &Condicionamiento \\
\hline
9&9.7358e-02&5.0092e+03\\
25&5.4641e-03& 7.4491e+08 \\
81& 5.4927e-06& 8.8868e+19\\
169&1.8050e-07&5.6865e+18\\
289& 1.5657e-07&2.8295e+19\\
1089& 7.2250e-09& 5.2878e+19\\
\hline
\end{tabular}
\end{center}
\end{table}
\begin{table}
\begin{center}
\caption{Segunda estrategia de variación del parámetro de forma. $\epsilon_j=\sqrt{\epsilon_{min}^2\frac{\epsilon_{max}^2}{\epsilon_{min}^2}^{\frac{j-1}{N-1}}}$}
\begin{tabular}{|c|cc|}
\hline
$N$ & ECM & Condicionamiento\\
\hline
9&2.0330e-01 & 2.6470e+02\\
25& 2.5906e-02&6.7274e+04\\
81&1.4758e-03&3.4960e+09 \\
169 &  4.8806e-04&2.9859e+14 \\
289 & 2.5673e-04&2.7370e+20\\
1089& 1.4820e-04& 1.3458e+20\\
\hline
\end{tabular}
\end{center}
\end{table}
\begin{table}
\begin{center}
\caption{RBF:IMQ.  EDP 3 (Wertz)r. Búsqueda del parámetro de forma óptimo (izquierda) y búsqueda del parámetro de forma óptimo en la frontera fijandolo en el interior (derecha).}
\begin{tabular}{|c|ccc|ccc|}
\hline 
N & $\epsilon$ optimo & ECM & Condicionamineto & $\epsilon$ frontera & ECM & Condicionamiento \\ 
\hline 
9 & 0.1 &6.8318e-01 & 3.4989e+09&0.1 &6.8318e-01   &3.4989e+09  \\ 
25 & 0.5& 1.1637e-01&1.7687e+09& 0.1&  6.3326e-02&1.1930e+14 \\
81&0.3& 1.5598e-03&1.4280e+19&0.1& 2.2413e-03 & 1.5289e+18  \\
169& 0.7&8.8125e-04& 3.0855e+18& 0.5& 2.5117e-05& 4.0733e+18\\
289&0.9& 2.1364e-05&6.0151e+18&0.5& 7.5085e-06& 1.0508e+20 \\
1089&2.1&1.0772e-05&2.2353e+20&1.3&5.6764e-07& 2.5768e+20 \\
\hline
\end{tabular} 
\label{EDP 3}
\end{center}
\end{table}


\begin{table}
\begin{center}
\caption{RBF: IMQ. EDP 3 (Wertz). Variación del parámetro en torno a la media fijada en la Tabla \ref{EDP 3}. $\epsilon_j=\bar{\epsilon} \frac{1}{\sqrt{1+K(-1)^j}}$. $K=0.3$ en el interior y $K=0.5$ en la frontera. }
\begin{tabular}{|c|cc|}
\hline
$N$ & ECM &Condicionamiento \\
\hline
9&2.0840e+00&1.1905e+08\\
25&1.0330e-01& 8.4037e+11 \\
81&5.8824e-05&2.2419e+19 \\
169&4.0977e-05& 5.6865e+18\\
289&2.9787e-05& 1.4668e+21 \\
1089&1.1837e-06 & 5.8059e+20\\
\hline
\end{tabular}
\end{center}
\end{table}
\end{document}